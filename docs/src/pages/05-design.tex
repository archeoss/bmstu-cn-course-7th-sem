\section{Конструкторский раздел}

\subsection{Обработка запросов HTTP}

Запрос, в контексте протокола HTTP, структурирован в виде последовательности строк, начиная с первой строки, содержащей информацию о методе, URL и версии HTTP. 
Последующие строки представляют заголовки запроса, которые служат для дополнительной спецификации параметров запроса. 
Кроме того, запрос может включать тело, позволяя передавать дополнительные данные для более сложных веб-запросов.

Пример запроса показан на листинге~\ref{lst:http}

\begin{center}
\captionsetup{justification=raggedright,singlelinecheck=off}
\begin{lstlisting}[label=lst:http,caption=HTTP запрос]
GET / HTTP/1.1
Host: localhost:8000
User-Agent: Mozilla/5.0 (X11; Linux x86_64; rv:122.0) Gecko/20100101 Firefox/122.0
Accept: text/html,application/xhtml+xml,application/xml;q=0.9,image/avif,image/webp,*/*;q=0.8
Accept-Language: en-US,en;q=0.5
Accept-Encoding: gzip, deflate, br
Connection: keep-alive
\end{lstlisting}
\end{center}
\clearpage

\subsection{Uniform Resource Locator}

URL (Uniform Resource Locator) представляет собой строку, используемую для определения местоположения ресурса в интернете. 
Он обычно состоит из нескольких компонентов, таких как схема, хост, порт, путь, запрос, и фрагмент.
Путь в URL указывает на конкретный ресурс на сервере. 
Это может быть директория или файл. 
Например, в URL <<https:\/\/example.com\/path\/to\/resource>>, <<\/path\/to\/resource>> --- это путь.
Строка запроса следует за путем и начинается с вопросительного знака <<?>>. 
Она содержит параметры запроса в виде пар <<ключ=значение>>, разделенных амперсандом <<\&>>. 
Например, в URL <<https:\/\/example.com\/search?q=query>>, <<q=query>> --- это строка запроса.
Фрагмент обозначается символом <<\#>> и используется для указания конкретной секции внутри ресурса. 
Он служит для навигации внутри самого документа.
Например, в URL <<https:\/\/example.com\/page\#section>>, <<section>> --- это фрагмент.
Также при обработке URL особую роль играет URL-кодировка. 
Она применяется для передачи специальных символов в URL без их конфликта с зарезервированными символами. 
Она заменяет определенные символы и пробелы специальными последовательностями. 
Например, пробел заменяется на <<.

Для хранения информации о запросе предлагается использовать структуру, содержащую следующие поля:
\begin{itemize}
 \item тип метода: GET, HEAD;
 \item относительный путь url;
 \item query параметр url;
 \item fragment параметр url;
 \item список заголовков запроса;
\end{itemize}

\subsection{Обработка ответов HTTP}

Протокол HTTP формирует ответ, включающий в себя версию HTTP, числовое представление статуса, текстовую строку статуса, перечень заголовков и тело ответа. 
При этом, содержанием ответа может являться как текстовый, так и бинарный файл. 
В случае изображения, которое необходимо разбить на несколько буферов для передачи клиенту, данная операция осуществляется для
обеспечения эффективной передачи данных, особенно когда размер изображения превышает объем одного буфера.
Для хранения информации об ответе предлагается использовать структуру, содержащую следующие поля:

\begin{enumerate}
  \item статус;
 \item относительный путь файла;
 \item путь файла в запросе;
 \item MIME тип ответа;
 \item список заголовков ответа;
\end{enumerate}

\subsection{Безопасность}

Особое внимание уделяется безопасности статического веб-сервера. 
Наибольшую угрозу представляет несанкционированный доступ к файлам, расположенным вне директории сервера. 
Для предотвращения такого несанкционированного доступа рекомендуется использовать механизм, известный как Chroot Jail. 
Этот механизм реализуется с использованием системного вызова chroot, который изменяет корневой каталог на указанный, таким образом, ограничивая доступ к файлам вне этой директории. 
Следует отметить, что для выполнения системного вызова chroot необходимы привилегии суперпользователя. 
После выполнения системного вызова происходит снижение привилегий до уровня обычного пользователя, лишенного прав для вызова chroot, чтобы обеспечить безопасный доступ к остальным директориям. 
Этот подход позволяет эффективно управлять безопасностью работы статического веб-сервера.

\subsection*{Вывод}

Были сформулированы конструкторские требования к безопасности, асинхронности, формату запросов и ответов, а также к структурам, используемым для моделирования работы сервера.

\clearpage
