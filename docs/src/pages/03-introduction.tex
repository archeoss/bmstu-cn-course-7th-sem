\specsection{ВВЕДЕНИЕ}
% \addcontentsline{toc}{specsection}{Введение}

В настоящее время компьютерные сети играют все более важную роль в нашей повседневной жизни. Статические серверы – одна из самых распространённых и актуальных технологий в области сетевых взаимодействий. Они позволяют обеспечить эффективную и стабильную работу веб-приложений, сайтов и других сервисов, предоставляя контент пользователям через сеть. В связи с быстрым развитием интернета и все возрастающим спросом на онлайн-сервисы, понимание основ и принципов работы статических серверов является существенным для специалистов в области сетевых технологий. 

В данной курсовой работе будет идти сосредоточиение на изучении статических серверов, их важности и показателях распространённости, а также на приобретении навыков настройки и обслуживания такой системы.

Для достижения поставленной цели, предполагается выполнение следующих задач:
\begin{enumerate}
    \item провести формализацию задачи и определить необходимый функционал;
    \item исследовать предметную область веб серверов;
    \item спроектировать приложение;
    \item реализовать приложение;
    \item протестировать приложение на предмет корректности;
\end{enumerate}
