\section{Исследовательский раздел}

\subsection{Технические характеристики}

Ниже приведены технические характеристики устройства, на котором будет проведено исследование:

\begin{itemize}
    \item[$-$] Операционная система: Arch Linux~\cite{arch-linux} 64-bit;
        \item[$-$] Количество ядре: 4 физических и 8 логических ядер;
    \item[$-$] Оперативная память: 16 Гб, DDR4;
    \item[$-$] Процессор: 11th Gen Intel\textsuperscript{\tiny\textregistered} Core\textsuperscript{\tiny\texttrademark} i5-11320H @ 3.20 ГГц~\cite{i5}.
\end{itemize}

Во время тестирования устройство было подключено к сети электропитания и было нагружено только встроенными приложениями окружения, а также непосредственно системой тестирования.

\subsection{Работа сервера}

На рисунке \ref{img:get_200} показан пример GET запроса.

\img{100mm}{get_200}{Демонстрация работы: GET запрос}

На рисунке \ref{img:head_200} показан пример HEAD запроса.

\img{80mm}{head_200}{Демонстрация работы: HEAD запрос}

На рисунке \ref{img:get_404} показан пример запроса несуществующего файла.

\img{80mm}{get_404}{Демонстрация работы: GET запрос к несуществующему файлу.}

\clearpage

На рисунке \ref{img:log} показан пример записи логгера.

\img{40mm}{log}{Демонстрация работы: Запись логгера}

% \clearpage

\subsection{Нагрузочное тестирование}

Нагрузочное тестирование проводилось с помощью ApacheBenchmark, 1000 запросов файла 27Мб в 8 потоков. 
В качестве альтернативы был настроен сервер nginx. 
Результаты нагрузочного тестирования представлены в таблице \ref{tab:bench}.

\begin{table}[h]
	\begin{center}
		\caption{Результат нагрузочного тестирования}
		\label{tab:bench}
		\begin{tabular}{|c|c|c|c|c|}
			\hline
			Сервер & Время & Запрос/с & Среднее время запроса & Объём передачи\\
			\hline
			nginx & 25.373с & 39.41 & 25.373мс & 1.092Гб/с \\ \hline
			Thr.Pool и Poll & 24.651с & 40.57 & 24.651мс & 1.112Гб/с \\ \hline
		\end{tabular}
	\end{center}
\end{table}

\clearpage

\subsection*{Вывод}

Проведено исследование и нагрузочное тестирование разработанного приложения и nginx, в ходе которого выявлено, что сервер nginx показал схожую производительность приведенной имплементации.

