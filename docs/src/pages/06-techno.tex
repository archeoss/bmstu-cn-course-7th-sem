\section{Технологический раздел}

\subsection{Листинги кода}

В листинге~\ref{lst:serv} представлена реализация запуска веб-сервера:

\begin{center}
\captionsetup{justification=raggedright,singlelinecheck=off}
\begin{lstlisting}[label=lst:serv,caption=Запуск WEB-сервера]
int main(int argc, char *argv[]) {
  ...
  /* defaults */
  size_t nthreads = 4;
  size_t nslots = 64;
  char *servedir = ".";

  rlim.rlim_cur = rlim.rlim_max =
      3 + nthreads + nthreads * nslots + 5 * nthreads;
	if (setrlimit(RLIMIT_NOFILE, &rlim) < 0) {
    die("setrlimit:");
  }
  insock = sock_get_ips(srv.host, srv.port);
  if (sock_set_nonblocking(insock)) {
    return 1;
  }
  /* chroot */
  if (chdir(servedir) < 0) {
    die("chdir '%s':", servedir);
  }
  if (chroot(".") < 0) {
    die("chroot:");
  }
  /* accept incoming connections */
  server_init_thread_pool(insock, nthreads, nslots, &srv);

	while (wait(&status) > 0)
    ;

  return status;
}
\end{lstlisting}
\end{center}

Функция обработки соединения, изображенная в листинге~\ref{lst:serv-accept}, включает в себя принятие соединения, поиск подходящего соединения для дропа, принятие заголовка, и парсинг заголовков.

\begin{center}
\captionsetup{justification=raggedright,singlelinecheck=off}
\begin{lstlisting}[label=lst:serv-accept,caption=Функция обработки соединения]
struct connection *connection_accept(int insock, struct connection *connection,
                                     size_t nslots) {
  struct connection *c = NULL;
  size_t i;

  for (i = 0; i < nslots; i++) {
    if (connection[i].fd == 0) {
      c = &connection[i];
      break;
    }
  }
  if (i == nslots) {
    c = connection_get_drop_candidate(connection, nslots);
    c->res.status = 0;
    connection_log(c);
    connection_reset(c);
  }

  if ((c->fd = accept(insock, (struct sockaddr *)&c->ia,
                      &(socklen_t){sizeof(c->ia)})) < 0) {
    if (errno != EAGAIN && errno != EWOULDBLOCK) {
      warn("accept:");
    }
    return NULL;
  }

  if (sock_set_nonblocking(c->fd)) {
    return NULL;
  }
  return c;
}
\end{lstlisting}
\end{center}

На листинге~\ref{lst:serv-drop} представлен поиск соединения для завершения, при новом соединении.


\begin{center}
\captionsetup{justification=raggedright,singlelinecheck=off}
\begin{lstlisting}[label=lst:serv-drop,caption=Функция поиска соединения]
enum status http_prepare_header_buf(const struct response *res,
                                    struct buffer *buf) {
  char tstmp[FIELD_MAX];
  size_t i;

  memset(buf, 0, sizeof(*buf));

  if (timestamp(tstmp, sizeof(tstmp), time(NULL))) {
    memset(buf, 0, sizeof(*buf));
    return S_INTERNAL_SERVER_ERROR;
  }
  if (buffer_appendf(buf,
                     "HTTP/1.1 %d %s\r\n"
                     "Date: %s\r\n"
                     "Connection: close\r\n",
                     res->status, status_str[res->status], tstmp)) {
    memset(buf, 0, sizeof(*buf));
    return S_INTERNAL_SERVER_ERROR;
  }
  for (i = 0; i < NUM_RES_FIELDS; i++) {
    if (res->field[i][0] != '\0' &&
        buffer_appendf(buf, "%s: %s\r\n", res_field_str[i], res->field[i])) {

      memset(buf, 0, sizeof(*buf));
      return S_INTERNAL_SERVER_ERROR;
    }
  }

  if (buffer_appendf(buf, "\r\n")) {
    memset(buf, 0, sizeof(*buf));
    return S_INTERNAL_SERVER_ERROR;
  }

  return 0;
}
\end{lstlisting}
\end{center}

\subsection{Демонстрация работы}

Для демонстрации корректности работы прогаммы был написан простой веб-сайт, состоящий из трех файлов: index.html, styles.css, README.md. 
На рисунке~\ref{img:work} показан веб-сайт открытый в браузере Firefox.

\img{160mm}{work}{Демонстрация работы}

Для демоснтрации работы сервера на больших файлах был использован образ установки ArchLinux, весящий 586 Мб и команда wget.

\img{60mm}{work_big}{Демонстрация работы с большими файлами}

\subsection*{Вывод}

Программа была реализована на языке Си, проверка на корректость пройденна успешно.

